\documentclass[12pt]{article}
\usepackage{indentfirst}
\usepackage{fancyhdr}
\usepackage[style=apa]{biblatex}

\addbibresource{sources.bib}

\pagestyle{fancy}

\fancyhf{}
\fancyhead[R]{Bontera Research, \thepage}
\fancyhead[L]{\leftmark}
\fancyfoot[C]{Hayaan Rizvi}
\setlength{\headheight}{15pt}
\renewcommand{\footrulewidth}{0.4pt}

\setcounter{tocdepth}{2}

\makeatletter
\def\@seccntformat#1{%
  \expandafter\ifx\csname c@#1\endcsname\c@section\else
  \csname the#1\endcsname\quad
  \fi}
\makeatother

\begin{document}

\addcontentsline{toc}{section}{Title}
\begin{center}
	\Huge\textbf{Bontera Research}\\

	\Large{Hayaan Rizvi}\\

	\Large{Dr. Imam}\\
	As of \today\\
	\hfill \break
	\noindent\rule{10cm}{0.4pt}

\end{center}

\setlength{\parindent}{0.5in}

\hfill \break
\newpage

\addcontentsline{toc}{section}{Contents}
\tableofcontents

\newpage
\section{Potatoes}
\subsection{Soft Rot}
\subsubsection{Description}

Bacterial soft rot affects a number of fruits and vegetables, including potatoes. It is a post-harvest disease, occuring while the crop is stored or in transit \autocite{rich2013potato}. It is characterized by a watery soft spot on the side of the crop and an strong odor.

\subsubsection{Geography and Soil Type}

Soft rot arises worldwide, from South Africa \autocite{ngadze2012pectinolytic} to the northeastern United States \autocite{ge2021genotyping}. Pérombelon notes that the soil has to be both nutritionaly deficient and over-watered for soft rot to spread. Some bacterium are specific to potato soft rot because of the cool, temperate climate where they are grown \autocite{perombelon2002potato}. Soft rot favors warm temperatures and high moisture levels in the soil and during storage \autocite{rich2013potato}.

\subsubsection{Cause}

Two main genera cause soft rot: \emph{Pectobacterium} and \emph{Dickeya} \autocite{youdkes2020potential}. Bacterial growth is accelerated by any wound or puncture of the potato skin, especially a tender spot formed by standing water or insect bites breaking the outer layer \autocite{rich2013potato}.

\subsubsection{Biosolution}

Historically, farmers have used water-management and sanitation to control soft rot, but the bacterial predator \emph{Bdellovibrio bacteriovorus} and similar organisms have shown considerable promise as well, according to a recent study by Youdkes et al. (2020). All the strains tested were effective in reducing \emph{Pectobacterium} and \emph{Dickeya} greatly. Strains introduced in the tubers before the onset of soft rot were much more effective in fighting it later on, possibly because glucose consumption by the prey was not inhibiting growth \autocite{youdkes2020potential}. A second study also determined that a mixture of \emph{Pseudomonas putida} and \emph{Pseudomonas fluorescens} decreased the severity of the disease and prevented transmission to child potatoes.

A third study found that certain varieties of rhizosphere bacteria could also be used to control soft rot caused by \emph{Pectobacterium} strains \autocite{krzyzanowska2012rhizosphere}. The results indicate that 18 various rhizobacteria were effective in inhibiting the spread of soft rot, out of the 1165 tested. The study noted, however, that the bacteria's ability to control \emph{Pectobacterium} \emph{in vitro} may not reflect the true results \emph{in vivo}.

\subsection{Golden Nematode Disease}
\subsubsection{Description}

Nematodes are nearly invisible worms that attack potato plants and tubers \autocite{rich2013potato}. Rich also reports that some nematodes cause disease directly, while others act as vectors or catalysts for viral and fungal illnesses. Plants affected by golden nematode disease have necrotic and wilting leaves with stunted growth; many do not recover, resulting in severely reduced crop yeilds \autocite{rich2013potato}.

\subsubsection{Geography and Soil Type}
The golden nematode is native to Peru, but has spread all throughout the world \autocite{rich2013potato}. They can survive in any climate that potatoes can grow, but very strict quarantine and sanitation guidelines have prevented further spread \autocite{evans1980origin}.

Mimee et al. (2015) found that optimal nematode egg hatching occurs when soil temperatures are between 59$^\circ F$ and 80$^\circ F$. The determined that increasing temperatures in cooler climates could lead to accelerated spread of golden nematode disease. Both dry weather and light soil favor the disease \autocite{rich2013potato}.

\subsubsection{Cause}

The golden nematode (\emph{Globodera rostochiensis}) is the main cause of Golden Nematode Disease \autocite{rich2013potato}. The disease gets its name from the golden or brown cysts containing nematode eggs present after an infestation \autocite{rich2013potato}. This is how the the disease spreads: the cysts cling to containers, equipment, and tubers and are transferred between fields or farms.

\subsubsection{Biosolution}

Some farmers have reported the natural decline of nematode populations because of fungi parasitism \autocite{evans1993reviews}. 10 species were isolated from the soil, \emph{Cylindrocarpon destructans} being the most promising \autocite{evans1993reviews}. That same study found that ``when an inoculum of straw colonised by \emph{C. destructans} was placed around potato seed tubers planted in [nematode] infested soil ... the numbers of juvenile stages of \emph{G. rostochiensis} ... decreased by 62\%.''


\subsection{Brown Rot}

\subsubsection{Description}
Brown rot, also known as bacterial wilt, is a very destructive bacterial disease \autocite{rich2013potato}. It was first reported it in the United States in 1896. Potatoes afflicted with brown rot excrete a ``slimy ooze'' at the base of the stem and eyes, although infected plants may also produce healthy tubers \autocite{rich2013potato}. In the early stages of the disease, the shoot system may wilt and the tubers will turn brown \autocite{kabeil2008potato}.

\subsubsection{Geography and Soil Type}
Brown rot heavily favors warmer climates; so much so that it is virtually eraticated in Canada and the northern United States \autocite{rich2013potato}. If it does infect potatoes in cooler climates, brown rot is harder to detect early on \autocite{kabeil2008potato}. The plants may not show any signs of infection until later stages of infection.

In a study by van Elsat et al. (2000), a steady decline of brown rot was observed in loamy sand soil. They also found that severe drought drastically reduced brown rot infections across the board. The disease prefers moist, slightly acid, netral, and alkaline soil \autocite{rich2013potato}.

\subsubsection{Cause}

\emph{Pseudomonas solanacearum} is the bacterium that causes brown rot \autocite{van2000survival}. There are three races, each of which attack different crops and have different optimal climates \autocite{rich2013potato}:

\textbf{Race 1}: Attacks eggplant, tobacco, tomato, and potatoes

\textbf{Race 2}: Attacks banana

\textbf{Race 3}: Attacks mainly potato, but is weakly pathogenic to tobacco

\emph{P. solanacearum} usually enters a host through a wound on the skin or roots \autocite{rich2013potato}. Rich (2013) also reports that the aflicted crops may infect the soil, transfering brown rot to any new plants grown afterwards.

\subsubsection{Biosolution}

One effective way to combat brown rot is to add sulfur to sandy soil, followed by limestone in the summer \autocite{rich2013potato}. Furthermore, a number of potato variants are resistant to \emph{P. solanacearum}; planting these cultivars will help minimize losses \autocite{rich2013potato}.

A study was also conducted by Fujiwara et al. in 2011 that experimented with using bacteriophages to control brown rot. While many of the viruses they tried were effective, application of {$\phi$}RSL1 (from family \emph{Myoviridae}) was the best method. Although it did not kill all the bacterial cells, it was successful in preventing the onset of disease. {$\phi$}RSL1 worked much better when it was introduced before \emph{P. solanacearum} as a preventative measure, although it did also have an effect when it was introduced after \autocite{fujiwara2011biocontrol}.

\subsection{Common Scab}

\subsubsection{Description}

Common scab is a fungal disease that shows no symptoms above ground in potato plants \autocite{rich2013potato}. While not effecting yeild, common scab makes the infected tuber less appealing to look at, effecting marketability to consumers \autocite{rich2013potato}. Symptoms include scabs and lesions of varying shapes and sizes on the skin of the tuber \autocite{dees2012search}.

\subsubsection{Geography and Soil Type}

Common scab occurs on every continent except Antarctica \autocite{rich2013potato}. In the United States, it is concentrated around the northwest, midwest, and northeastern portions \autocite{braun2017potato}. The disease prefers ``a pH
higher than 5.2, temperatures of 20–22 °C, and a soil moisture
below field capacity during early tuberization.'' \autocite{braun2017potato, archuleta1981cause}

\subsubsection{Cause}

The main cause of common scab in potatoes is various species of \emph{Streptomyces} \autocite{flores2008detection}. It is disputed whether \emph{Streptomyces} is a fungus or bacterium because it demonstrates characteristics of both (the USDA considers it a fungus) \autocite{rich2013potato}. The responsible strains synthesize thaxtomins and phytotoxins, chemicals required for the scabs to form \autocite{flores2008detection}. Flores-Gonz{\'a}lez et al. (2008) also report that the toxins exhibit properties similar to those of an infection.

\subsubsection{Biosolution}

The usual technique for controlling common scab has been seed treatment with pentachloronitrobenzene or maneb-zinc dust, but these show minimal effect on certain causal species \autocite{lee2004vivo}. Therefore, a more effective solution is needed for the control of common scab. Four \emph{Streptomycete} isolates, A020645, A010321, A010564, and A020973, showed antagonistic behavior against the common scab in a study by Lee et al. (2004). They lowered the formation of scabs and lesions by \textgreater 60\% and 55\%, respectively. Additionally, these four isolates were also resistant against 10 commonly used chemicals in potato fields, further improving their effectiveness.


\section{Corn (Maize)}

\subsection{Dry Rot}

\subsubsection{Description}

Dry rot is the most common ear rot disease \autocite{ullstrup1961corn}. Corn infected early in the growing season becomes grayish brown with fungal growth in between the kernels \autocite{ullstrup1961corn}. If the ear is infected later, then it shows no superficial signs, the only symptom being a white mold forming in the cob and slightly discolored kernels \autocite{ullstrup1961corn}.

\subsubsection{Geography and Soil Type}

Dry rot can occur at anywhere from 40° to 90°F, the best temperature being about 80°F \autocite{melhus1922dry}. The disease favors a dry spell in the summer followed by wet weather while the corn silks \autocite{ullstrup1961corn, melhus1922dry}. The optimal soil pH for dry rot growth is between 3.7 and 5.9, with a small amount of growth occuring in slightly basic soil \autocite{eddins1930dry}. Dry rot is especially widespread in soil with low potassium and high nitrogen \autocite{ullstrup1961corn}.

\subsubsection{Cause}

Scientists determined in 1834 that the primary cause of dry rot is the fungus \emph{Stenocarpella maydis} \autocite{durrell1923dry, ullstrup1961corn}. Mature spores are carried by the wind in warm, moist weather to healthy plants \autocite{ullstrup1961corn}. It enters into the ear through wounds in the husk, but may also infect the stalk and roots. Spores can lie dormant in soil for potentially several years while waiting for a host to infect \autocite{eddins1930dry}. \emph{S. maydis} infects the center of the ear first, only coming to the surface to release its spores \autocite{melhus1922dry}. Some hybrid species of corn have mild resistance, but no species is completely immune \autocite{ullstrup1961corn}.

\subsubsection{Biosolution}

Two \emph{Streptomyces} isolates, DAUFPE 11470 and DAUFPE 14632, were very effective in the control of \emph{S. maydis}; they reduced the pathogen in seedlings by 87.3\% and 85.6\%, respectively \autocite{bressan2005biological, bressan2003biological}. Seed germination was also increased by about 30\% when \emph{Streptomyces} was introduced in the same study.

Furthermore, several strains of \emph{Bacillus subtilis} were found to be very effective in the control of \emph{S. maydis}, as well as able to withstand extreme temperature and pH values \autocite{petatan2011isolation}. They inhibited dry rot by nearly 100\%. In a recent study by Petat{\'a}n-Sagah{\'o}n et al. (2011), it was reported that ``the strains of \emph{B. subtilis} are the most promising due to their ability to produce endospores that can remain
in the soil during long periods and can be produced easily at an industrial level with a low cost \autocite{stein2005bacillus}.''


\subsection{Stewart's Bacterial Wilt}

\subsubsection{Description}

Bacterial wilt is a bacterial illness that can very quickly lead to the loss of sweet corn \autocite{robert1967bacterial}. Infected plants resemble corn with a low water supply and may not produce ears if they survive the disease \autocite{ullstrup1961corn}. When the stalks are cut, yellow cysts of bacteria ooze out from the wound \autocite{ullstrup1961corn}. Streaks of dead tissue form on the leaves before they wilt \autocite{robert1967bacterial}. The disease is much less deadly on varieties of dent corn then it is on sweet corn \autocite{rand1933bacterial}.

\subsubsection{Geography and Soil Type}

The eastern United States experiences the worst effects of bacterial wilt in corn \autocite{ullstrup1961corn, robert1967bacterial}. While one harsh winter is enough to reduce the disease in the north, a series of cold winters is needed further south \autocite{robert1967bacterial}. However, multiple mild winters can build up the disease greatly \autocite{robert1967bacterial}. Soil type has little effect on bacterial wilt, although plants are more likely to die in unfertile soil \autocite{ullstrup1961corn, thomas1924stewart}. Thomas reported in 1924 that, while there was evidence of infection through seeds, no plants were infected through the soil in a greenhouse study.

\subsubsection{Cause}

\emph{Erwinia stewartii} is the cause of Stewart's bacterial wilt in corn \autocite{ullstrup1961corn}. Ullstrup (1961) found that \emph{E. stewartii} is carried through the winter on the bodies of corn flea beetles: ``they feed on young corn plants and start infections on the leaves ... [and] spread the disease from infected to healthy plants.'' Streaks on the leaves of infected plants start from the beetle feeding on it, and then spread along the veins and up into the stalk \autocite{robert1967bacterial}. Lesions along the stalk of the corn plant can be covered with fungi, which is often mistaken to be the cause of the wilting and infection \autocite{ullstrup1961corn}.

\subsubsection{Biosolution}

Wicklow and Poling found in 2009 that \emph{Acremonium zeae} produces natural antibiotics that can protect corn against a number of pathogens that cause bacterial wilt, stalk rot, and leef blight in corn. It was, however, ineffective against \emph{E. stewartii} \autocite{wicklow2009antimicrobial}. Currently, the best control against Stewart's bacterial wilt is the cultivation of resistant hybrids (Golden Cross Bantam, Country Gentleman, Evergreen, etc.) and the use of insecticides to kill beetles that carry the disease \autocite{ullstrup1961corn}. Correct amounts of potassium minimize bacterial wilt, while high nitrogen predispose corn to infection \autocite{ullstrup1961corn}.


\subsection{Northern Corn Leaf Blight (NCLB)}

\subsubsection{Description}

This leaf disease is a very devastating fungal illness \autocite{zhang2021klebsiella}. If it becomes well established a couple weeks following silking, crop yeild can be reduced by over 30\% \autocite{ullstrup1961corn}. Symptoms include long spots on leaves that spread until the entire plant (sans kernels) is taken over late in the growing season \autocite{ullstrup1961corn}. The symptoms resemble those of frost damage once northern corn leaf blight has progressed to a late stage.

\subsubsection{Geography and Soil Type}

NCBL is found in most moist environments where corn can be grown \autocite{ullstrup1961corn}. When conditions are perfect, such as in Florida, the disease can appear very early in the growing season and infect seedlings \autocite{ullstrup1961corn}. Ullstrup reports in his book, \emph{Corn Diseases in the United States and Their Control}, that ``germination and penetration take place within 6 to 18 hours when free water is present and the temperature is within 65° and 80°F.'' He found that it is prevalent in the eastern half of the United States, extending into the central area when conditions are favorable.

\subsubsection{Cause}

NCBL is caused by the heterothallic fungus \emph{Setosphaeria turcica} \autocite{leonard1989proposed}. It overwinters on corn leaves until its spores form on old lesions \autocite{ullstrup1961corn}. Then the wind can carry it a great distance during the summer. The races of \emph{S. turcica} are clasified based on their virulence against specific corn genes:

\textbf{Race 0}: All \emph{Ht} genes are resistant to it

\textbf{Race 1}: Can attack \emph{Ht1}

\textbf{Race 23}: Can attack \emph{Ht23}

\subsubsection{Biosolution}

\emph{Klebsiella jilinsis} 2N3 is a bacterium that can degrade some hebicides, but also promotes corn growth and NCLB resistance \autocite{zhang2021klebsiella}. Treated plants showed a 67.44\% decrease in \emph{S. turcica} in pot experiments, as well as increased production of defence enzymes \autocite{zhang2021klebsiella}. Furthermore, Zhang et al. determined in 2021 that \emph{K. jilinsis} 2N3 can ``grow in a nitrogen-free environment, dissolve inorganic phosphorus and potassium, and produce indoleacetic acid and a siderophore.''

\newpage


\addcontentsline{toc}{section}{References}
\printbibliography

\end{document}
